Экзопланеты почти никогда не находятся там, где они родились.

Когда мы смотрим на каталоги экзопланет, перед нами — плотные системы,
где планеты обращаются очень близко к своим звёздам,
часто с периодами всего в несколько дней.

Но эти орбиты — не место рождения.
Это финальное состояние сложной и длительной эволюции.

Чтобы понять, где на самом деле формируются планеты,
нужно выйти за рамки отдельных систем
и рассматривать их в галактическом контексте.

\vspace{0.5em}

Начнём с металличности — содержания тяжёлых элементов в звёздах-хостах.

На Рисунке~1 показано распределение металличности звёзд,
у которых обнаружены экзопланеты.
Выборка охватывает широкий диапазон значений,
но заметно смещена в сторону высоких металличностей.

Это не случайность.
Именно тяжёлые элементы формируют твёрдые ядра планет.

\vspace{0.5em}

Прежде чем делать физические выводы,
необходимо проверить селекционные эффекты.

На Рисунке~2 показана металличность как функция расстояния до системы.
Здесь нет физического градиента —
мы видим лишь пределы наблюдательных возможностей.

Экзопланеты легче обнаружить у близких,
ярких и, как правило, более металличных звёзд.

Это подтверждает, что сама металличность
не является артефактом наблюдений.

\vspace{0.5em}

Далее рассмотрим свойства самих звёзд.

Рисунки~3 и~4 показывают зависимость металличности
от эффективной температуры и поверхностной гравитации.
Большинство звёзд-хостов находятся на главной последовательности,
и металличность слабо зависит от их эволюционного состояния.

Это важный вывод:
металличность не формируется в процессе эволюции звезды,
она отражает условия среды,
в которой звезда родилась.

\vspace{0.5em}

Свойства планет усиливают эту картину.

На Рисунке~5 видно,
что массивные газовые гиганты
предпочтительно обращаются вокруг металличных звёзд.
Планеты меньшей массы встречаются
практически при любых значениях металличности.

Аналогичная картина наблюдается и для радиусов планет
(Рисунок~6):
крупные планеты сконцентрированы при высоких металличностях,
в то время как малые распространены гораздо шире.

Эти зависимости лежат в основе модели аккреции ядра.
Для формирования массивных планет
необходимо достаточное количество твёрдого вещества,
а оно напрямую связано с металличностью.

\vspace{0.5em}

Наконец, Рисунок~7 показывает динамическую сторону вопроса.

Зависимость массы планеты от орбитального периода
распадается на несколько характерных популяций.
Горячие юпитеры занимают короткопериодические орбиты,
в то время как на больших периодах
встречается широкий диапазон масс.

Объяснить такую структуру
исключительно формированием на месте практически невозможно.
Она указывает на орбитальную миграцию.

Планеты перемещаются.
Их текущие орбиты
не совпадают с местами их рождения.

\vspace{0.5em}

Все эти результаты подводят нас к ключевому вопросу.

Если металличность отражает условия рождения звезды,
а планеты после формирования мигрируют,
то определяющим параметром становится
не текущее положение планеты,
а то, где в Галактике сформировалась её звезда.

Это и есть идея радиуса рождения планет.

Чтобы по-настоящему понять формирование планет,
необходимо восстановить орбиты звёзд,
проследить их движение в диске Млечного Пути
и связать планетные системы
с химической эволюцией Галактики.

Именно этим мы займёмся дальше.
