Planets rarely form where we observe them today.

When we look at exoplanet catalogues, we see planets tightly packed around their stars, often on short-period orbits. 
But these positions are deceptive. They are not birthplaces — they are the final outcome of a long dynamical history.

To understand where planets truly form, we must look beyond individual systems and consider the broader galactic context.

\vspace{0.5em}

We begin with metallicity — the abundance of heavy elements in host stars.

Figure~1 shows the metallicity distribution of stars hosting known exoplanets. 
The sample spans a wide range of metallicities, but it is clearly biased toward metal-rich systems.
This is not a coincidence. Heavy elements are the building blocks of planetary cores.

\vspace{0.5em}

Before drawing physical conclusions, however, we must check for observational biases.

Figure~2 shows host-star metallicity as a function of distance.
There is no intrinsic metallicity gradient here — only the limits of our observations.
We detect planets more easily around nearby, bright, metal-rich stars.

This confirms that metallicity itself is not an observational artifact.

\vspace{0.5em}

Next, we examine stellar parameters.

Figures~3 and~4 show metallicity as a function of effective temperature and surface gravity.
The majority of host stars lie on the main sequence, and metallicity does not strongly depend on stellar temperature or evolutionary stage.

This tells us something important:
metallicity is not produced later in stellar evolution — it is an imprint of the environment in which the star was born.

\vspace{0.5em}

Planetary properties reinforce this picture.

Figure~5 shows a clear trend: massive gas giants preferentially orbit metal-rich stars.
Lower-mass planets, by contrast, are found across almost the entire metallicity range.

Figure~6 shows a similar pattern for planet radius.
Large planets cluster at higher metallicities, while smaller planets appear nearly everywhere.

These trends are a cornerstone of the core-accretion model.
Forming massive planets requires solid material — and that material depends on metallicity.

\vspace{0.5em}

Finally, Figure~7 reveals the dynamical side of the story.

Planet mass versus orbital period shows distinct populations.
Hot Jupiters occupy short-period orbits, while longer periods host a wide range of planet masses.

This structure is difficult to explain through in-situ formation alone.
It strongly suggests orbital migration.

Planets move.
Their present-day orbits are not their birth locations.

\vspace{0.5em}

Taken together, these results point to a deeper question.

If metallicity reflects the conditions of star formation,
and if planets migrate after they form,
then the key quantity is not where a planet is today,
but where its host star was born in the Galaxy.

This is the idea of the planetary birth radius.

To understand planet formation, we must reconstruct stellar orbits,
trace stars back through the Milky Way,
and connect planetary systems to the chemical evolution of the Galactic disk.

That is the next step in this journey.
